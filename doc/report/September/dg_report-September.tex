\documentclass[11pt, a4paper]{article}
\usepackage{amsmath}
\usepackage{amsfonts}
\usepackage{amssymb}
\usepackage{geometry}
\usepackage{authblk}
\usepackage{placeins}

\usepackage{graphicx}
\usepackage{epstopdf}
\usepackage{subcaption}
\usepackage{color}
\usepackage{wasysym}
\newcommand{\rred}[0]{\color{red}}
\newcommand{\rblue}[0]{\color{blue}}

% Neater hyperlinks
\usepackage{hyperref}
\hypersetup{
   breaklinks,
   pdfcreator    = \LaTeX{},
   pdfproducer   = \LaTeX{},
   bookmarksopen = true,
   colorlinks = true,
   urlcolor   = cyan,
   citecolor  = red,
   linkcolor  = black,
   filecolor  = blue,
}

% Tensorial and finite volume packages
\usepackage{general}
\usepackage{tensorCommon}
\usepackage{tensorEquation}
\usepackage{tensorOperator}
\usepackage{tabularx}
\newcolumntype{C}[1]{>{\centering\let\newline\\\arraybackslash\hspace{0pt}}m{#1}}
\newcolumntype{Y}{>{\centering\arraybackslash}X}
\newcolumntype{s}{>{\hsize=.5\hsize}Y}
\usepackage{multirow}\usepackage{finiteVolume}

\usepackage[center]{titlesec}
\usepackage{fancyhdr}

% For multicolumn tables


% Left justify section headings
\usepackage{sectsty}
\sectionfont{\bfseries\Large\raggedright}

% Hide subsubsections in table of contents
\setcounter{tocdepth}{2}

\title{\vspace{4cm}%
\Huge
\bf Discontinuous Galerkin implementation in foam-extend\\[2cm]
\huge
Report-1\\
September 2018\\[1cm]
}

\author[]{Gregor Cvijeti\'{c}}
\affil[]{ \small University of Zagreb, Croatia, gregor.cvijetic@fsb.hr}

\renewcommand\Authands{ and }

\newgeometry{left=1in, right=1in, top=1in, bottom=1in}
\date{\vspace{-5ex}}

%\pagestyle{empty}
\pagestyle{fancy}

\begin{document}

\maketitle
\thispagestyle{empty}

\pagebreak

\tableofcontents



\section*{Summary}
\label{sec:1}

\noindent In this report the implementation of discontinuous Galerkin (DG)
method in
foam-extend is presented. As reports will be delivered on a monthly basis, of
which this is the first one, the fundamental programming and mathematical basis,
as well as the core code structure are presented here. Each following report
will continue from where the previous report left off, without repeating parts
that are not in the focus of the current work. Therefore, all of the reports
should be considered as a whole, rather than each report as an independent
document.\\

\noindent The report contains:
\begin{enumerate}
    \item The mathematical basis for discontinuous Galerkin method,
    \item The mathematical model for Poisson equation,
    \item Discussion on the code structure,
    \item Details of the implementation and numerical procedures.
\end{enumerate}

\vspace{12pt}

\noindent In the section \nameref{sec:2} an overview of the implemented
mathematical approach will be given. Furthermore, the Gauss integration and
choice of modal basis will be briefly covered. In the section \nameref{sec:3}
the code structure will be presented. As the code structure adheres to the
foam-extend standard, only the overview of the complete code will be given,
while parts of the code that are specific to DG will be highlighted.  Section
\nameref{sec:4} presents a walkthrough of important implementation parts
needed for code understanding and getting familiar with how the solver
works.


%%%%%%%%%%%%%%%%%%%%%%%%%%%%%%%%%%%%%%%%%%%%%%%%%%%%%%%%%%%%%%%%%%%%%%%%%%%%%%%
%%%%%%%%%%%%%%%%%%%%%%%%%%%%%%%%%%%%%%%%%%%%%%%%%%%%%%%%%%%%%%%%%%%%%%%%%%%%%%%

\section{Mathematical model}
\label{sec:2}

In this section the mathematical model of the discontinuous Galerkin method will
be presented. No theory will be covered as it is accessible
in a number of sources \ref{refereirati kummer lecture notes, phd, onu moju
knjigo}, therefore only equations and relevant parts of theory will be covered.



\subsection{Modal basis}

As opposed to available alternative (nodal basis), a modal basis approach has
been chosen in this work. This means that referent cell value is calculated
as a superposition of modes (polynomials), whereas only the polynomial
coefficients (weights on modal values) $\varphi_i$ are stored:

\begin{equation}
    u(x) = \sum_{i=1}^N \varphi_i P_i(x) \mcomma
    \label{eq:1}
\end{equation}

\noindent where $N$ represents number of modes used, $P_i$ is a polynomial of
mode $i$ evaluated at coordinate $x$ and $\varphi_i$ is a polynomial coefficient. The
equation \eqref{eq:1} is valid cell-wise, but at this point refers only to a
referent cell, meaning that $x$ refers to a local coordinate, $x\in\{-1,1\}$.

The choice of polynomials is arbitrary, while due to benefitial properties
\ref{neka, kummer} the Legendre polynomials are used here:
%
\begin{align}
    P_0 &= 1\notag\\
    P_1 &= x\notag\\
    P_2 &= \dfrac{1}{2}(3x^2 - 1)\\
    P_3 &= \dfrac{1}{2}(5x^2 - 3x)\notag\\
    P_4 &= \dfrac{1}{8}(35x^4 - 30X^2 + 3)\notag\\
    &...\notag
\label{eq:2}
\end{align}

or using the recursive formula:

\begin{equation}
    P_{n+1}(x) = \dfrac{2n+1}{n+1}xP_n(x) - \dfrac{n}{n+1} P_{n-1}(x)
    \label{eq:3}
\end{equation}

with

\begin{equation}
    P_0(x) = 1 \textrm{ and }  P_1(x) = x \mfstop
\label{eq:4}
\end{equation}


\noindent For the initial work only modes up to second order are implemented and
the implementation for 1-D problems is performed.


\subsection{Gaussian quadrature integration}

Gaussian quadrature rule refers to approximation of the definite integral of a
function, formulated as a weighted sum of function values at specified points
within the domain of integration:

\begin{equation}
    \int_{-1}^{1} f(x) dx \approx \sum_{i=1}^{n} w_i f(x_i) \mcomma
\label{eq:5}
\end{equation}

Points $x_i$ at which the function is evaluated are called Gauss nodes.
The accuracy of the approximation depends on the number of used points, whereas
in case of Legendre polynomials of the order $2n-1$, $n$ Gauss nodes are needed
to obtain the exact value of the integral.
Weighting factors $w_i$ are calculated using the relation:

\begin{equation}
    w_i = \dfrac{2}{(1-x_i^2)[P_n'(x_i)]^2}
\label{eq:6}
\end{equation}

\noindent while coordinate $x_i$ is the $i$-th root of the polynomial $P_n$. The
quadrature rule is used for evaluating the volume and surface integrals
throughout the code.



TREBAM LI PROVJERITI DA LI SE INTEGRAL RACUNA SAMO U PAR TOCAKA ILI BI SVAKI
POLINOM TREBAO U SVOJIM TOCKAMA




\subsection{Poisson equation}

First task of the implementation is to solve the Poisson equation. The
discretisation is performed using the Symmetric Interior Penalty (SIP)
method, with the aid of the mean value operator $\{\cdot\}$ and jump operator
$[\![\cdot ]\!]$ on the cell edges, due to the discontinuous nature of cell-wise
polynomials. Following \cite{Kummer} the discretized form of Poisson equation
multiplied by a test function states:

\begin{equation}
    \int_\Omega \div (\grad u) v dV = \int_\Omega \grad u \cdot \grad v dV
    -\oint_\Gamma \{\grad u\} \cdot n_\Gamma [\![v]\!] dA
    -\oint_\Gamma \{\grad v\} \cdot n_\Gamma [\![u]\!] dA
    +\oint_\Gamma \eta [\![u]\!] [\![v]\!] dA
\label{eq:7}
\end{equation}

\noindent where on the right hand side of the equation terms from left to right
are: volume term, consistency term, symmetry term and penalty term. Jump and
mean value operators are:

\begin{align}
    \{c_h\} &= \dfrac{c_h^- + c_h^+}{2}\\
    [\![c_h]\!] &= c_h^- - c_h^+
\label{eq:8}
\end{align}

where $c_h$ is a value within the cell $h$, $-$ sign represents internal (or
cell side) value on the edge, while $+$ represents value on neighbouring cell
side of the edge.

TU BI BILO DOBRO RASPISATI MATRICE I SVE


\section{Code structure}
\label{sec:3}

In this section a code structure will be briefly presented. Although most of the
parts will be covered, a special attention will be given only to parts that are
significantly different from common foam-extend structure or to parts that don't
have its alternative in the finite volume library.

\subsection{Source code}

\subsection{Unit tests}

\subsection{Solvers}



\section{Implementation details}
\label{sec:4}

\subsection{Primitives}

\subsection{Mesh}

\subsection{Fields}

\subsection{Matrix}
\subsection{Operators}
\subsection{Boundary Conditions}

Napisati da ce biti provedena usporedba s Bosss code i Kummer's polynomials.

Osvrnuti se na numerical flux

%%%%%%%%%%%%%%%%%%%%%%%%%%%%%%%%%%%%%%%%%%%%%%%%%%%%%%%%%%%%%%%%%%%%%%%%%%%%%%%
%%%%%%%%%%%%%%%%%%%%%%%%%%%%%%%%%%%%%%%%%%%%%%%%%%%%%%%%%%%%%%%%%%%%%%%%%%%%%%%

\pagebreak

\bibliographystyle{plain}
\bibliography{%
}%

\end{document}
